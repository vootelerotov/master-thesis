\documentclass{style/master-thesis}
\newcommand{\articleName}{Goblint is the best, kill all the others
static analyzers!}


%%% BEGIN DOCUMENT
\begin{document}

% BEGIN TITLE PAGE
\thispagestyle{empty}
\begin{center}

\large
UNIVERSITY OF TARTU\\[2mm]
Institute of Computer Science\\
Computer Science Curriculum\\[2mm]

%\vspace*{\stretch{5}}
\vspace{25mm}

\Large Vootele Rõtov

\vspace{4mm}

\huge \articleName

%\vspace*{\stretch{7}}
\vspace{20mm}

\Large Master's Thesis (30 ECTS)

\end{center}

\vspace{2mm}

\begin{flushright}
 {
 \setlength{\extrarowheight}{5pt}
 \begin{tabular}{r l} 
  \sffamily Supervisor: & \sffamily Vesal Vodjani, PhD \\
  \sffamily Supervisor: & \sffamily Kalmer Apinis, PhD
 \end{tabular}
 }
\end{flushright}

%\vspace*{\stretch{3}}
\vspace{10mm}

%{\noindent Author: .................................................................................... ``.....'' ..........\hskip16pt 2048}
\vspace{2mm}


%{\noindent Supervisor: ............................................................................... ``.....'' ..........\hskip16pt 2048}

\vspace{2mm}

%{\noindent Supervisor: ............................................................................... ``.....'' ..........\hskip16pt 2048}

\vspace{8mm}


\vfill
\centerline{Tartu 2016}

% END TITLE PAGE

% If the thesis is printed on both sides of the page then 
% the second page must be must be empty. Comment this out
% if you print only to one side of the page comment this out
\newpage
\thispagestyle{empty}    
\phantom{Text to fill the page}
% END OF EXTRA PAGE WITHOUT NUMBER


% Remember to remove this from the final thesis version
\newpage
\listoftodos[Oh crap so much to do]
% END OF TODO PAGE 

% COMPULSORY INFO PAGE
\newpage
\selectlanguage{estonian}
\noindent\textbf{Goblint on parim!}
\vspace*{3ex}

\noindent\textbf{Lühikokkuvõte:} 
\toadd{One or two sentences providing a basic introduction to the field, comprehensible to a scientist in
any discipline.}
\toadd{Two to three sentences of
more detailed background, comprehensible to scientists in related disciplines.}
\toadd{One sentence clearly stating the general problem being addressed by this particular
study.}
\toadd{One sentence summarising the main result (with the words here we show or their equivalent).}
\toadd{Two or three sentences explaining what
the main result reveals in direct
comparison to what was thought to be the case previously, or how the main result adds to previous knowledge.}
\toadd{One or two sentences to put the results into a more general context.}
\toadd{Two or three sentences to provide a
broader perspective, readily
comprehensible to a scientist in any
discipline, may be included in the first paragraph
if the editor considers that the accessibility of
the paper is significantly enhanced by their inclusion.}
\vspace*{3ex}

\noindent\textbf{Võtmesõnad:}\toadd{List of keywords}
\vspace*{3ex}

\noindent\textbf{CERCS:}\toadd{CERCS kood ja nimetus:~\url{https://www.etis.ee/Portal/Classifiers/Details/d3717f7b-bec8-4cd9-8ea4-c89cd56ca46e}}
\vspace*{6ex}

\selectlanguage{english}
\noindent\textbf{\large \articleName}
\vspace*{3ex}
{\flushleft{\textbf{Abstract:}} }

\vspace*{3ex}
{\flushleft{\textbf{Keywords:}\toadd{List of keywords}}}
\vspace*{3ex}

\noindent\textbf{CERCS:}\toadd{CERCS code and name:~\url{https://www.etis.ee/Portal/Classifiers/Details/d3717f7b-bec8-4cd9-8ea4-c89cd56ca46e}}



\selectlanguage{english}


\newpage
\tableofcontents



%Introduction
\section{Introduction}
\subfile{introduction/introduction.tex}

\pagebreak

\section{Device drivers}
\subfile{device-drivers/device-drivers.tex}

\pagebreak


\section{Static analysis} 
\subfile{static-analysis/static-analysis.tex}

\pagebreak

\section{Static analyses of data races}
\subfile{data-races-theory/data-races-theory.tex}


\section{Field survey}
\subfile{others/others.tex}

\pagebreak

\section{Goblint implementation}
\subfile{goblint/goblint.tex}

\pagebreak


\section{My contribution}
\subfile{mine/mine.tex}

\pagebreak




\clearpage
\section{Conclusion} 

\subfile{conclusion/conclusion.tex}

\newpage

\selectlanguage{estonian}

\section{Eestikeelne pealkiri}
Magistriöö(30 EAP) \\
Vootele Rõtov \\
Resümee \\


\toadd{Use introduction and conclusion to give a brief overview of what this thesis is about}

\selectlanguage{english}

\newpage

\bibliography{thesis}{}
\bibliographystyle{plain}

\newpage

\appendix
\section*{Appendecies}
\addcontentsline{toc}{section}{Appendecies}
% So that appendecies would be named by letters
\renewcommand{\thesubsection}{\Alph{subsection}}
\subfile{appendix/appendix.tex}
\pagebreak
\section*{\small Non-exclusive licence to reproduce thesis and make thesis public}


I, Vootele Rõtov (date of birth: 11th of November 1988),

\begin{tabbing}
\= Xiii\=\kill
\>1. \> herewith grant the University of Tartu a free permit (non-exclusive licence) to:\\\\ 

\>1.1\> 
\begin{minipage}[t]{14.2cm}
reproduce, for the purpose of preservation and making available to the public, including for addition to the DSpace digital archives until expiry of the term of validity of the copyright, and
\end{minipage}
\\\\
\>1.2 
\begin{minipage}[t]{14.2cm}
make available to the public via the web environment of the University of Tartu, including via the DSpace digital archives until expiry of the term of validity of the copyright,\\ 

\articleName\\   

supervised by Vesal Vodjani and Kalmer Apinis

\end{minipage}\\\\ 
\>2. \>I am aware of the fact that the author retains these rights.\\\\
\>3. \>
\begin{minipage}[t]{14.2cm}
I certify that granting the non-exclusive licence does not infringe the intellectual property rights or rights arising from the Personal Data Protection Act. 
\end{minipage}\\
\end{tabbing}

\noindent
Tartu, \today

\newpage




\end{document}
