\documentclass{style/master-thesis}
\newcommand{\articleName}{Goblint is the best, kill all the others
static analyzers!}


%%% BEGIN DOCUMENT
\begin{document}

% BEGIN TITLE PAGE
\thispagestyle{empty}
\begin{center}

\large
UNIVERSITY OF TARTU\\[2mm]
Institute of Computer Science\\
Computer Science Curriculum\\[2mm]

%\vspace*{\stretch{5}}
\vspace{25mm}

\Large Vootele Rõtov

\vspace{4mm}

\huge \articleName

%\vspace*{\stretch{7}}
\vspace{20mm}

\Large Master's Thesis (30 ECTS)

\end{center}

\vspace{2mm}

\begin{flushright}
 {
 \setlength{\extrarowheight}{5pt}
 \begin{tabular}{r l} 
  \sffamily Supervisor: & \sffamily Vesal Vodjani, PhD \\
  \sffamily Supervisor: & \sffamily Kalmer Apinis, PhD
 \end{tabular}
 }
\end{flushright}

%\vspace*{\stretch{3}}
\vspace{10mm}

%{\noindent Author: .................................................................................... ``.....'' ..........\hskip16pt 2048}
\vspace{2mm}


%{\noindent Supervisor: ............................................................................... ``.....'' ..........\hskip16pt 2048}

\vspace{2mm}

%{\noindent Supervisor: ............................................................................... ``.....'' ..........\hskip16pt 2048}

\vspace{8mm}


\vfill
\centerline{Tartu 2016}

% END TITLE PAGE

% Remember to remove this from the final thesis version
%\pagebreak
%\listoftodos[Oh crap so much to do]
% END OF TODO PAGE 

% COMPULSORY INFO PAGE
\pagebreak

\selectlanguage{english}
\noindent\textbf{\large \articleName}
\vspace*{3ex}
\begin{flushleft}
  \textbf{Abstract:} The concurrent nature of device drivers makes them notoriously difficult to manually debug.
  Goblint, a static analysis framework tries to automatically verify the inexistence of data races. 
  The key challenge in doing that is the precision of the analysis.
  This paper proposes an enhancement to the region analysis of Goblint to incorporate domain specific happens-before guarantees. 
  The proposed addition is implemented and evaluated on the Goblint benchmark suite.
  We show that the given enhancement increases the precision of Goblint when analysing character drivers.
\end{flushleft}


\vspace*{3ex}
\begin{flushleft}
  \textbf{Keywords:} Concurrency, race condition, data race, abstract interpretation, happens-before, device drivers, Goblint
\end{flushleft}
\vspace*{3ex}

\noindent\textbf{CERCS:} P170, Computer science, numerical analysis, systems, control
\selectlanguage{estonian}

\vspace*{5ex}
\noindent\textbf{\large \articleNameEE}
\vspace*{3ex}

\begin{flushleft}
  \textbf{Lühikokkuvõte:} Seadmedraiverite ehk ohjurite paralleelne olemus muudab nendest vigade leidmise inimese jaoks väga keeruliseks.
  Staatiline analüsaator Goblint üritab automaatselt verifitseerida, et ohjuris puuduvad andmejooksud. Sealjuures on suureks väljakutseks analüüsi täpsus.
  Käesolev töö arendab edasi Goblinti regioonianalüüsi, mis võimaldab  arvesse võtta valdkonnale eriomaseid \textit{happens-before} tagatisi.
  Väljapakutud täienduse implementeerimise ning muudatuste mõju analüüsimise aluseks on võetud Linuxi ohjurite alamhulk.
  Me näitame, et mainitud edasiarendus suurendab Goblinti täpsust \textit{character} tüüpi ohjurite analüüsimisel.
\end{flushleft}
\vspace*{3ex}

\begin{flushleft}
  \textbf{Võtmesõnad:} Paralleelsus, andmejooks, abstraktne interpretatsioon, seadmedraiverid, ohjurid, Goblint
\end{flushleft}
\vspace*{3ex}

\noindent\textbf{CERCS:} P170, Arvutiteadus, arvutusmeetodid, süsteemid, juhtimine (automaatjuhtimisteooria)

\newpage

\selectlanguage{english}



\tableofcontents



%Introduction
\section{Introduction}
\subfile{introduction/introduction.tex}

\pagebreak

\section{Device Drivers}
\label{sec:device-drivers}
\subfile{device-drivers/device-drivers.tex}

\pagebreak


\section{Theoretical Foundations of Static Analysis}
\label{sec:static-analysis}
\subfile{static-analysis/static-analysis.tex}

\pagebreak

\section{Data Race Analysis}
\label{sec:data-races}
\subfile{data-races-theory/data-races-theory.tex}

\subsection{Field Survey}
\subfile{others/others.tex}

\pagebreak


%\section{Implementation in Goblint}
%\subfile{goblint/goblint.tex}

\section{Region Analysis of Space and Time}
\label{sec:region}
\subfile{space-time-region/space-time-region.tex}

\pagebreak


\section{Implementation in Goblint}
\label{sec:goblint}
%\subfile{mine/mine.tex}
\subfile{goblint/goblint.tex}


\pagebreak


\clearpage
\section{Conclusion} 

\subfile{conclusion/conclusion.tex}

\newpage

\bibliography{thesis}{}
\bibliographystyle{plain}

\newpage

\appendix
\section*{Appendecies}
\addcontentsline{toc}{section}{Appendecies}
% So that appendecies would be named by letters
\renewcommand{\thesubsection}{\Alph{subsection}}
\subfile{appendix/appendix.tex}
\pagebreak
\section*{\small Non-exclusive licence to reproduce thesis and make thesis public}


I, Vootele Rõtov (date of birth: 11th of November 1988),

\begin{tabbing}
\= Xiii\=\kill
\>1. \> herewith grant the University of Tartu a free permit (non-exclusive licence) to:\\\\ 

\>1.1\> 
\begin{minipage}[t]{14.2cm}
reproduce, for the purpose of preservation and making available to the public, including for addition to the DSpace digital archives until expiry of the term of validity of the copyright, and
\end{minipage}
\\\\
\>1.2 
\begin{minipage}[t]{14.2cm}
make available to the public via the web environment of the University of Tartu, including via the DSpace digital archives until expiry of the term of validity of the copyright,\\ 

\articleName\\   

supervised by Vesal Vodjani and Kalmer Apinis

\end{minipage}\\\\ 
\>2. \>I am aware of the fact that the author retains these rights.\\\\
\>3. \>
\begin{minipage}[t]{14.2cm}
I certify that granting the non-exclusive licence does not infringe the intellectual property rights or rights arising from the Personal Data Protection Act. 
\end{minipage}\\
\end{tabbing}

\noindent
Tartu, \today


\end{document}
