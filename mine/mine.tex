\documentclass[..thesis.tex]{subfiles}



\begin{document}

\toguide{Okay, what's next?}



\toguide{Neat. Is this all of any use?}

Evaluating this approach on the set of character devices used as benchmark suit for Goblint gave following result.

\toadd{Actual benchmark}

As seen by the results of the benchmark, the effect this partitioning has depends heavily on the driver.
In most of the drivers the effect can be attributed to the exclusion of races between init and exit functions. 

\toask{Should mention this?} 

It is worth noting that the motivation for introducing \textit{happens-before} based analysis to Goblint did came from analyzing the common issues on a subset of those very same drivers.
However, the fact that there were other drivers, outside the subset of the benchmark drivers that was manually analyzed,
that greatly benefited from the addition of the extra dimension leads the author to believe that there is a reasonable likelihood
that the benefits are not limited to this specific set of drivers.

\toguide{Okay, got it. Seems cool. Anything else we should know?}

The examples given for guarantees were all specific to Linux device drivers and as such, required domain knowledge to specify. This is an hindarance,
specially taking into account that the number of possible coventions that guarantee race freedom between two operations is very likely large.  

However, at the same time it is very hard to imagine a domain independent analysis that would be able to safely exclude the races that we have looked at in this section.

Currently, the rules mentioned in this section are hard-coded into Goblint. This approach does not completely solve the issue of how to make Goblint take the domain specific happen-before
guarantees into account -- sadly it is not possible for the team behind Goblint to cover all the possible domains nor even fully cover the domain of device drivers.

In the long term, offering either the user of a Goblint or an interested thirty party a way to specify those guarantees,
for example through a DSL,  would help to solve this limitation. In addition, an approach that would try to deduce possible happens-before guarantees dynamically 
could be something that would lessen the burden on the party specifying the guarantees.

\toadd{Some kind of ending}
\tosup{It would be nice to have some words about how this is implemented in slightly more low-level terms.}




\end{document}
