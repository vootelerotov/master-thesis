
\documentclass[..thesis.tex]{subfiles}
\begin{document}

\toguide{What is static analysis?}
Under \textit{program analysis} (or just \textit{analysis}) we mean a process of deciding whether some property holds for a program under analysis.
Analysis of a program is \textit{static} if the program being analysed does not get executed during the analysis, in contrast to \textit{dynamic} analysis,
during what program is executed. Static analysis are often used by compilers, for example for finding uses of variables that are not declared beforehand
in Java or C programs. In addition, modern IDEs constantly run wide selection of static analysis in the background, to enable things such as variable renaming,
for what one would have to know about all the usages of a specific variable in the program, taking into account the relevant scopes. 

\toguide{Why limit ourselves?}
There are good reasons for preferring static analyses over dynamic (or combined) analysis.
To pick a few, it lets us say something about a program that does not compile (and as such, cannot be analyzed dynamically),
it allows to perform our analyses faster compared to dynamic analysis what one cannot perform faster than the time it takes to execute the program
and as static analyses do not depend use any information gathered from a specific run then it is possible to have an analysis that says something
about all the possible runs of a program (and not only, say, for all the possible runs that the analysis has witnessed).
The last reason is, as one can imagine, very helpful in case of data-races. This enables static analysis to output to find that there
 are no errors in the program instead of that it is not able to find any errors -- to verify a program instead of testing it.

\toguide{How is static analysis done}?
As one might guess, there are a lot of techniques that fit the very wide definition given for static analysis. We will be focusing on one of them -- \textit{abstract interpretation}.

\toguide{Okay, what is abstract interpretation?}
To give an intuation on what is abstract interpretation, consider following program:

\begin{lstlisting}[language=C,style=def]
int main(void) {
  int y,z;
  y = -81;
  z = -37
  if (y * z > 0){
    return 1;
  }
  return 0;
}
\end{lstlisting}.

One can easily deduce that the program will always exit with error code $1$, without having to calculate the exact value of  $y*z$.
Instead, we can intepret all the values variables have as negative ($-$), $0$ or positive ($+$) and evaluate $y * z$ as $-*- = +$.
This is considerable easier task.

With this simplification, we do lose presicion. Lets consider another example:

\begin{lstlisting}[language=C,style=def]
int main(void) {
  int y,z;
  y = -81;
  z = -37
  if (y * z > 0){
    return 1;
  }
  if (y * z - y > 0){
    return 2;
  }
  return 0;
}
\end{lstlisting}.

Here, using the interpretation what we succesfully applied in previous example, we would not be able to decide the return value of the program. 
When evaluating $y * z - y $, we can rule out any of the possible values. To be able to evaluate product in a similar way previously described,
we could assigs variables set of \textit{possible} values instead. In this case, we would evaluate $y * z - y $ as $\left\lbrace - \right\rbrace * \left\lbrace
- \right\rbrace - \left\lbrace - \right\rbrace = \left\lbrace -,0,+ \right\rbrace$. Although we cannot decide what the program will return,
we can decide based on the analysis, that the first $if$ block can safely be left out -- an optimization a compiler could perform.

\toguide{Is it useful if you cannot even evaluate this simple program?}








  


\textit{abstract domain} --$D$,  a set of states for abstract interpretation. Partially ordered $\left(D,\sqsubseteq \right)$ such that if $n \in N$, $d_1,d_2 \in D$,  $n \in \gamma \left(d_1 \right) $ and $d_1 \sqsubseteq d_2$ then $n \in \gamma \left(d_2 \right)$.

\textit{CFG} -- $G_p=\left(N_p,E_p,s_p,r_p\right)$, $E_p \subseteq N_p \times N_p$, where $N_p$ is set of all states of program, $E_p$ set of state tranformations, $s_p$ the starting state of a procedure and $r_p$ the end state of a procedure. $N$ is union of $N_p$ over all procedures.

\textit{abstract function} --$\alpha$,  mapping from $N$ to $D$.

\textit{concreitization function} -- $\gamma$, mapping from $D$ to $2^N$.


\textit{context sensitive} -- takes into account the values of parameters of function calls, analyses for each call site vs something that applies for all the call sites.

\textit{complete latice} -- a partially ordered set $\left( D,\sqsubseteq \right)$ is complete lattice if for every $S \subseteq D$ there is a an upper and lower bound and if there exists a greatest and a . We denote uppor bound of set $S$ as $\bigsqcup S $ and lower bound as $\bigsqcap S$. We denote the least element of a lattice ($\bigsqcap D$) as $\bot$ and the greatest element as $\bigsqcup D$ as $\top$. Note that we by $d_1 \bigsqcup d_2$ we mean  $\bigsqcup \left\lbrace d_1, d_2, \right\rbrace $   and by $ d_1 \bigsqcap d_2$ we mean $\bigsqcap \left\lbrace d_1, d_2, \right\rbrace $.

 

\textit{Merge over all paths} -- 



\end{document}