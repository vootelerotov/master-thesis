%!TEX root = ../thesis.tex
\documentclass[..thesis.tex]{subfiles}

\begin{document}

\kalmer{On veel tükk tööd teha, enne kui saab lihvimisele mõelda. Mõned üldised märkused: \\ 1) sa oled lasknud end nakatada Vesali rumalatest naljadest (a la Goblint is the best, kill all the others static analyzers!) Tegelikult ei võistle me eriti kellegiga ja me ei pea ennast nii palju õigustama ja teiste najal upitama. Eriti magistritöö raames pole mingit probleemi kui oleksime Locksmithist rangelt kehvemad.\\
2) Punase niidi teooria. Jutustava teksti juures on vaja, et läbi kogu töö oleks selge narratiiv ja jutt ei hüppaks ilma peata ringi. See on ka üks põhjus, miks sa kohati üleliigset teksti kirjutad --- sul pole eesmärki silme ees ning sa ei saa kirjutada seda punast niiti, mis lugeja eesmärgini viib.
}

\todisc{Vaatasin üle related worki osa ja mulle tundub see suhteliselt neutraalne, ma ei võrdle ühtegi teist tööriista otseselt Goblintiga. Töö pealkiri on hetkel selline, mul oli placeholderit vaja ning kui ma selle repo lõin siin oli see vist Goblinti paberi kokkuvõtte. Kahjuks mul on nõrkus running joke'ide osas :). Kui ma midagi ei märganud, siis anna märku.}

\todisc{Punase niidi osas olen nõus. Proovisin kirjutades vastata küsimustele, mis minul kui lugejal võiks tekkida -- see lähenemine muutis kirjutamise mõnevõrra lihtsamaks aga kindlasti nõrgendas keskset narratiivi. }

\toguide{What is it in simple terms (title)?}
\toguide{Why should anyone care?}
\toguide{What was my contribution?} 
\toguide{What you are doing in each section (a sentence or two per section)}

Inspired by the happens-before relation and motivated by the results of benchmarking done for the article.

\end{document}