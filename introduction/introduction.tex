%!TEX root = ../thesis.tex
\documentclass[..thesis.tex]{subfiles}

\begin{document}

\toguide{What is it in simple terms (title)?}
 

\toguide{Why should anyone care?}

Device drivers are open programs that work in difficult environment. One of the difficulties is that device drivers have no control over when the callbacks they provide to the Linux kernel
are called. This, combined with the fact that device drivers are written in a subset of C, a low level language with very few built-in concurrency abstractions,
means that debugging the concurrency issues in device drivers is notoriously difficult. Empirical research confirms that concurrency bugs are common in
device drivers \cite{chou_empirical_2001,palix_faults_2011}.  
In addition to that, these bugs have a big impact and contribute to a large share of system crashes \cite{swift_improving_2003}. 

\toadd{analyser for -> analyser of}

Goblint is a sound static analyser for detecting data races in Linux device drivers, developed in the University of Tartu and the Technical University of Munich. Goblint detects a subtype of concurrency issues called data races -- situations where more than one thread simultaneously tries to access a shared memory location.
If Goblint as a sound tool does not find any data races, it has verified that the driver under analysis is free of data races.The key challenge for Goblint is to assure that the results of analysis would not only be sound, but would also contain as few false-positives as possible. High precision results either in outright verification of the driver or provides the user with valuable feedback. Low precision, however, produces too much noise to be useful. The aim of this thesis is to increase the precision of Goblint by eliminating one type of false-positives. These were found while manually reviewing the data races detected in the Goblint benchmark suite as part of my contribution to a research paper \cite{vojdani_static_2016}.
  
\toguide{What was my contribution?} 

As part of this thesis, an analysis performed by Goblint was enhanced with time dimension. Inspired by the happens-before relation and motivated by the results of the performed benchmarking,
the extra dimension enables Goblint to take into account guarantees provided by the Linux kernel of the way the exposed callbacks are invoked.
For example, let there be an assignment to memory location \inlinecode{i} in both of the functions \inlinecode{init} and \inlinecode{exit}. 
If we know that both functions are called only once and that call to \inlinecode{init} must complete before \inlinecode{exit} can be called then we can safely rule out a data race on \inlinecode{i}.
As a result of the added dimension, the analysis of most device drivers in benchmark suite are more precise. For 6 out of 25 drivers in the benchmark suite,
the results have improved notably.


\toguide{What you are doing in each section (a sentence or two per section)}
The thesis continues with Section \ref{sec:device-drivers}, where we give an overview of Linux device drivers and also show the extent of impact of bugs on them.
Next, in Section \ref{sec:static-analysis} we introduce reader to the basics of abstract interpretation, a static analysis technique Goblint uses.
Section \ref{sec:data-races} introduces two key concepts for static analysis of data races.
In addition, an overview is given of other notable tools for data race detection in device drivers. An overview of region analysis and theoretical background of my contribution, i.e.
time based partitioning, is given in Section \ref{sec:region}. Lastly, in Section \ref{sec:goblint}, we give an overview of Goblint,
the implementation details of the added enhancement and also the evaluation of effect that the changes had.  


\end{document}