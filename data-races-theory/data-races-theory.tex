\documentclass[..thesis.tex]{subfiles}

\begin{document}

As discussed previously, it is possible that the same function of a driver is executed concurrently.
Let \textit{thread template} be the executable code reachable from one of the registered entry points of the driver.
There can be any number of threads executing a single thread template at any point, corresponding to multiple calls to the same function of a device driver.

\subsection{Lockset analysis}

Lockset analysis is a high-level idea about how to verify the absence of data races in a program.
It relies on the assumption that the access to a shared variable should be governed by a lock.

Let us have two threads, $T_1$ and $T_2$ executing thread templates $t_1$ and $t_2$. Let's note that $t_1$ and $t_2$ do not have to be distinct.

Let $S$ be a shared state between those two threads,
containing variables that are accessible from both threads and $L$ be a set of all locks that can be held by either thread.

Now, let $O_{T}^v$ be a set of all read or write operations of variable $v \in S$ in thread $T$.
Let's define operation $\mathword{locks} : O_{T}^v \to 2^L$, that for a read or write operation of variable $v$ in $T$ returns a set of all the locks that thread $T$
holds when that operation is performed.

If set

\begin{equation*}
\left( \bigcap \limits_{o \in O_{T_1}^v} \mathword{locks} \left(o \right) \right) \cap  \left( \bigcap \limits_{o \in O_{T_2}^v} \mathword{locks} \left(o \right) \right)
\end{equation*}

is not empty, then it means that there exists $l \in L$ such that every read or write operation of variable $v$ in threads $T_1$ and $T_2$ is protected by $l$ and as such,
there is no chance for a data race. If such a lock exists for every variable $v \in S$, we can soundly say that there is no possibility of a data race between threads $T_1$ and
$T_2$ executing thread templates $t_1$ and $t_2$. Note that this approach can be easily generalized for $n$ threads and $m$ thread templates.

Let's note that although this analysis is sound (we will not verify a program that has a possibility of a data race),
it is not precise, as there might be other ways than locking to make sure that no data race can take place. This idea was first applied in \cite{engler_racerx:_2003}.


\subsection{Happens-before}

\textit{Happens-before} relation $R$ is a partial-ordering of the statements, such that if $a$ and $b$ are statements then $a,b \in R$ if and only if $a$ is executed before $b$.
The concept was first introduced by Leslie Lamport \cite{lamport_time_1978}, under the name happened-before.

For $a$ to be executed before $b$, either $a$ has to be a statement before $b$ in the same thread template,
there must be a statement $c$ such that $a$ is executed before $c$ and $c$ is executed before $b$ or
there must be a synchronization event that is sufficient to say that $a$ was executed before $b$.

As an example of a synchronization event, let's consider spawing a thread. Let $t_1$ and $t_2$ be two thread templates.
Let $t_1$ be a thread template where a variable $x$ is read and then a thread from template $t_2$ is spawned.
Let $t_2$ be such a thread template where a value is assigned to $x$. The thread spawning functions here as a synchronization event -- 
if the thread $T_1$ runs the template $t_1$ and spawns the thread $T_2$, then the read of $x$ in $t_1$ happens before the assignment in $t_2$.

The key insight concerning data races is that if we can establish happens-before relationship between statements $a$ and $b$, then they cannot race against each other. 

\end{document}